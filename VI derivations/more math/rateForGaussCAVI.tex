\documentclass{article}

\usepackage[a4paper, total={6in, 8.5in}]{geometry}
\usepackage[utf8]{inputenc}
\usepackage[T1]{fontenc}    
\usepackage{hyperref} 
\usepackage{url}   
\usepackage{booktabs}    
\usepackage{amssymb,amsfonts,amsmath,graphicx}      
\usepackage{nicefrac}       
\usepackage{microtype}    
\usepackage{breqn}
\usepackage[usenames, dvipsnames]{color}

\newcommand{\ind}[1]{1_{#1}} % Indicator function
\newcommand{\pr}{P} % Generic probability
\newcommand{\ex}{E} % Generic expectation
\newcommand{\var}{\textrm{Var}}
\newcommand{\cov}{\textrm{Cov}}
\newcommand{\sgn}{\textrm{sgn}}
\newcommand{\sign}{\textrm{sign}}
\newcommand{\kl}{\textrm{KL}} 
\newcommand{\abs}[1]{|{#1}|}



\renewcommand{\S}{\Sigma}
\renewcommand{\L}{\Lambda}
\renewcommand{\[}{\begin{equation}}
\renewcommand{\]}{\end{equation}}
\renewcommand{\b}{\backslash}
\newcommand{\g}{\,\vert\,}
\newcommand{\tr}{\mathrm{tr}}
\newcommand{\diag}{\mathrm{diag}}
\newcommand{\bea}{\begin{eqnarray}}
\newcommand{\eea}{\end{eqnarray}}
\newcommand{\hx}{\hat{x}}
\newcommand{\hxi}{\hat{\xi}}
\newcommand{\Var}{\mathrm{Var}}
\newcommand{\Cov}{\mathrm{Cov}}
\newcommand{\prop}{\propto}
\newcommand{\deq}{:=}

\newcommand{\EE}{\mathbb{E}}
\newcommand{\II}{\mathbb{I}}
\newcommand{\R}{\mathbb{R}}
\newcommand{\PP}{\mathbb{P}}

\newcommand{\La}{\mathcal{L}}

\newcommand{\n}{\mathcal{N}}

\newcommand{\bx}{\mathbf{x}}
\newcommand{\bX}{\mathbf{X}}
\newcommand{\by}{\mathbf{y}}
\newcommand{\bs}{\mathbf{s}}
\newcommand{\bn}{\mathbf{n}}
\newcommand{\br}{\mathbf{r}}
\newcommand{\bt}{\mathbf{t}}

\newcommand{\fig}[1]{Figure~\ref{fig:#1}}
\newcommand{\chap}[1]{Chapter~\ref{chap:#1}}
\newcommand{\mysec}[1]{Section~\ref{sec:#1}}
\newcommand{\app}[1]{Appendix~\ref{sec:#1}}
\newcommand{\eq}[1]{Eq.~(\ref{eq:#1})}
\newcommand{\eqs}[1]{Eqs.~(\ref{eq:#1})}
\newcommand{\eqss}[1]{(\ref{eq:#1})}
\newcommand{\thm}[1]{Theorem~\ref{thm:#1}}

\newcommand{\indep}{{\;\bot\!\!\!\!\!\!\bot\;}}
\newcommand{\eps}{\varepsilon}

\newcommand{\one}{1}
\newcommand{\Dir}{{\rm Dir}}
\newcommand{\Mult}{{\rm Mult}}
\newcommand{\Bin}{{\rm Bin}}
\newcommand{\Ga}{{\rm Ga}}
\newcommand{\IG}{{\rm IG}}
\newcommand{\InvGa}{{\rm IG}}
\newcommand{\Chisquare}{\Chi^2}
\newcommand{\St}{{\rm St}}
\newcommand{\Beta}{{\rm Beta}}
\newcommand{\iid}{i.i.d.}
\newcommand{\Eta}{{\cal N}}
\newcommand{\Ber}{{\rm Ber}}

\newcommand{\simiid}{\stackrel{\tiny\text{iid}}{\sim}}
\newcommand{\simind}{\stackrel{\tiny\text{ind}}{\sim}}

\DeclareMathOperator*{\BP}{BP}
\DeclareMathOperator*{\DP}{DP}
\DeclareMathOperator*{\GP}{GP}
\DeclareMathOperator*{\BeP}{BeP}

% Caligraphic alphabet
\newcommand{\calr}{\mathcal{R}} % only because \cr already taken
\newcommand{\ca}{\mathcal{A}} \newcommand{\cb}{\mathcal{B}} \newcommand{\cc}{\mathcal{C}} \newcommand{\cd}{\mathcal{D}} \newcommand{\ce}{\mathcal{E}} \newcommand{\cf}{\mathcal{F}} \newcommand{\cg}{\mathcal{G}} \newcommand{\ch}{\mathcal{H}} \newcommand{\ci}{\mathcal{I}} \newcommand{\cj}{\mathcal{J}} \newcommand{\ck}{\mathcal{K}} \newcommand{\cl}{\mathcal{L}} \newcommand{\cm}{\mathcal{M}} \newcommand{\cn}{\mathcal{N}} \newcommand{\co}{\mathcal{O}} \newcommand{\cp}{\mathcal{P}} \newcommand{\cq}{\mathcal{Q}} \newcommand{\cs}{\mathcal{S}} \newcommand{\ct}{\mathcal{T}} \newcommand{\cu}{\mathcal{U}} \newcommand{\cv}{\mathcal{V}} \newcommand{\cw}{\mathcal{W}} \newcommand{\cx}{\mathcal{X}} \newcommand{\cy}{\mathcal{Y}} \newcommand{\cz}{\mathcal{Z}}

% Convergence
\newcommand{\convd}{\stackrel{d}{\longrightarrow}} % convergence in distribution/law/measure
\newcommand{\convp}{\stackrel{P}{\longrightarrow}} % convergence in probability
\newcommand{\convas}{\stackrel{\textrm{a.s.}}{\longrightarrow}} % convergence almost surely
\newcommand{\convr}{\stackrel{r}{\longrightarrow}} % convergence in r^{th} mean

\newcommand{\eqd}{\stackrel{d}{=}} % equal in distribution/law/measure
\newcommand{\argmax}{\mathop{\mathrm{argmax}}}
\newcommand{\argmin}{\mathop{\mathrm{argmin}}}
\newcommand{\conv}{\textrm{conv}} % for denoting the convex hull


\makeatletter
\providecommand*{\diff}%
	{\@ifnextchar^{\DIfF}{\DIfF^{}}}
\def\DIfF^#1{%
	\mathop{\mathrm{\mathstrut d}}%
		\nolimits^{#1}\gobblespace}
\def\gobblespace{%
	\futurelet\diffarg\opspace}
\def\opspace{%
	\let\DiffSpace\!%
	\ifx\diffarg(%
		\let\DiffSpace\relax
	\else
		\ifx\diffarg[%
			\let\DiffSpace\relax
	\else
		\ifx\diffarg\{%
			\let\DiffSpace\relax
		\fi\fi\fi\DiffSpace}


\providecommand*{\deriv}[3][]{\frac{\diff^{#1}#2}{\diff #3^{#1}}}
\providecommand*{\pderiv}[3][]{\frac{\partial^{#1}#2}{\partial #3^{#1}}}
		
\newcommand{\threequals}{\equiv}

\DeclareMathOperator*{\argminU}{arg\,min}
\DeclareMathOperator*{\argmaxU}{arg\,max}


\begin{document}


\section{Gaussian Model} 

We consider a model in which $x\to y\to z$ forms a Markov chain, $p(x)\propto 1$ has a flat prior, and each conditional distribution is Gaussian. It follows that the posterior distribution has form
\begin{align}
p(x,y\mid z) \triangleq \cn(\mu^*,\Sigma),
\end{align}
where $\mu^* = (\mu_x^*,\mu_y^*)^{\mathsf T}$ and $\Sigma$ has blocks $\Sigma_{xx}, \Sigma_{xy},$ and $\Sigma_{yy}$, and these means and covariances depend on the observation $z$. From Barber [eqn. 8.6.11] we know the conditionals have the form 
\begin{align}
p(x\mid z,y) &= \cn(x; \mu_x^*+\Sigma_{xy}\Sigma_{yy}^{-1}(y-\mu_y^*),\Sigma_{xx} - \Sigma_{xy}\Sigma_{yy}^{-1}\Sigma_{yx}), \\
p(y\mid z,x) &= \cn(y; \mu_y^*+\Sigma_{yx}\Sigma_{xx}^{-1}(x-\mu_x^*),\Sigma_{yy} - \Sigma_{yx}\Sigma_{xx}^{-1}\Sigma_{xy}).
\end{align}
This gives the form of the CAVI variational factors 
\begin{align}
q^{(t+1)}(x) &= \cn(x; \mu_x^*+\Sigma_{xy}\Sigma_{yy}^{-1}(\widehat \mu_y^{(t)}-\mu_y^*),\Sigma_{xx} - \Sigma_{xy}\Sigma_{yy}^{-1}\Sigma_{yx}), \\
q^{(t+1)}(y) &= \cn(y; \mu_y^*+\Sigma_{yx}\Sigma_{xx}^{-1}(\widehat\mu_x^{(t)}-\mu_x^*),\Sigma_{yy} - \Sigma_{yx}\Sigma_{xx}^{-1}\Sigma_{xy}).
\end{align}
Where $\widehat\mu^{(t)}$ are the variational means at the $t$-th iteration. Hence 
\begin{align}
\widehat\mu_x^{(t+1)}-\mu_x^*
&= \Sigma_{xy}\Sigma_{yy}^{-1}(\widehat \mu_y^{(t)}-\mu_y^*)
= \Sigma_{xy}\Sigma_{yy}^{-1}\Sigma_{yx}\Sigma_{xx}^{-1}(\widehat\mu_x^{(t)}-\mu_x^*) \\
\widehat\mu_y^{(t+1)}-\mu_y^*
&= \Sigma_{yx}\Sigma_{xx}^{-1}\Sigma_{xy}\Sigma_{yy}^{-1}(\widehat\mu_y^{(t)}-\mu_y^*)
\end{align}
Note that if $\lambda$ is an operator of one of these matrices, i.e. $\Sigma_{yx}\Sigma_{xx}^{-1}\Sigma_{xy}\Sigma_{yy}^{-1} v = \lambda v$, then 
$$
\Sigma_{xy}\Sigma_{yy}^{-1}\Sigma_{yx}\Sigma_{xx}^{-1}(\Sigma_{xy}\Sigma_{yy}^{-1} v) = \lambda (\Sigma_{xy}\Sigma_{yy}^{-1}v),
$$
and similarly in the other direction, so the spectra of these matrices coincide. Let $\gamma = \|\Sigma_{xy}\Sigma_{yy}^{-1}\Sigma_{yx}\Sigma_{xx}^{-1}\|_2 = \|\Sigma_{yx}\Sigma_{xx}^{-1}\Sigma_{xy}\Sigma_{yy}^{-1}\|_2$, so the rate of convergence of each variational mean is
\begin{align}
\left\|\widehat\mu_x^{(t+1)}-\mu_x^*\right\|_2
&\le \gamma\left\|\widehat\mu_x^{(t)}-\mu_x^*\right\|_2 \\
\left\|\widehat\mu_y^{(t+1)}-\mu_y^*\right\|_2
&\le \gamma\left\|\widehat\mu_y^{(t)}-\mu_y^*\right\|_2
\end{align}
Hence the rate of convergence for the whole algorithm is $\gamma$, since
\begin{align}
\left\|(\widehat\mu_x^{(t+1)},\widehat\mu_y^{(t+1)})-(\mu_x^*,\mu_y^*)\right\|_2^2
&=\left\|\widehat\mu_x^{(t+1)}-\mu_x^*\right\|_2^2 + \left\|\widehat\mu_y^{(t+1)}-\mu_y^*\right\|_2^2\\
&\le \gamma^2\left\|\widehat\mu_x^{(t)}-\mu_x^*\right\|_2^2 + \gamma^2\left\|\widehat\mu_y^{(t)}-\mu_y^*\right\|_2^2 \\
&= \gamma^2\left\|(\widehat\mu_x^{(t)},\widehat\mu_y^{(t)})-(\mu_x^*,\mu_y^*)\right\|_2^2 
\end{align}
This rate $\gamma$ matches the rate of convergence of the corresponding block Gibbs sampler [Sahu \& Roberts, 1998, theorem 1].



\end{document}